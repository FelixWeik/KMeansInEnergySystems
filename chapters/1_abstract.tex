\chapter{Abstract}
\label{cha:abstract}
Climate change is one of the most pressing issues of our time. 
Researchers in this area are focused on finding solutions to reduce its impact.
One solution can be found in the energy sector, where a large part of the greenhouse gas emissions is produced.
This work focuses on reviewing the application of the k-means clustering algorithm to understand energy load patterns and thus improve the design of energy-saving incentives.
K-means clustering is one of the most popular and widely used clustering algorithms due to its simplicity and efficiency in handling large datasets.
The different papers demonstrate k-means' applications from different perspectives, such as identifying home systems of best practices or reviewing energy efficiency metrics of Chinese industry sectors.
The results show, that k-means works well in finding trends and patterns while handling large datasets efficiently.
This generates knowledge out of the data, which can be used as input for higher-level models or decision-making processes.
Also, k-means weaknesses in this area of usage are analyzed, namely the need for a priori knowledge or the missing ability to spot single outliers in the data.
The algorithm is efficient in its computational complexity, however, multiple executions are needed to evade the risk of local optima.
The research shows that k-means can indeed be used to understand energy load patterns and to improve the design of energy saving incentives.