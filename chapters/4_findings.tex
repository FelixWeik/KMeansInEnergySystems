\chapter{Findings}
\label{cha:findings}
% Grading Criteria
% - Steps taken to arrive at the presented findings 
% - The research question is addressed
% - present findings
% - graphs and table support the main findings 
% - findings are presented without judgment

This section presents the findings of the papers introduced in \autoref{subsec:k_means_in_the_context_of_energy_saving_incentives}.
The differences and similarities are juxtaposed for both research questions.

% Wie möchte ich die Research Questions erklären?
% Es werden nicht die Fragen beantwortet, sondern vielmehr eine Hinführung zur Conclusion gegeben,
% Also nicht auf die Conclusion eingehen, sondern auf die Results der jeweiligen Paper
% - Zusammenhängend => Mehr ein Fließtext, der die Fragen mithilfe der Paper beantwortet
% - Jedes Ergebnis soll so erklärt werden, dass es in den jeweiligen Kontext hineinpasst
% - Dabei alles sauber labeln und möglichst viel zitieren
\section{Improving the Design of Energy-Saving Incentives}
\label{sec:improving_the_design_of_energy_saving_incentives}

TODO: Delete the paragraphs and summarize both findings in continuous text
\paragraph*{LIU-BDE:}
% Wie soll dieser Paragraph strukturiert sein?
% 1. Allgemeine Ergebnisse ansprechen
% 2. Auf ein Beispiel eingehen (in diesem Fall das Clustering und die Tabelle)
The results are separated into four parts: Clustering for freshwater consumption, the environmental performance measured on the sulfur dioxide emissions, and the energy efficiency performance measured on the coal consumption.
The k-means algorithm is applied multiple times for each part: once for the whole dataset containing all industry sectors and for each industry sector separately.
Therefore, distinct companies can be assigned to a cluster and be compared with other companies from the same or different industry sectors.
One of the clustering algorithm's results is shown in \autoref{fig:multi_industries_clustering_result_environemental_performance}.
A certain threshold most companies remain below can be determined.
Furthermore, the number of companies in each cluster can be counted and compared.
The results of the assignments for each industry sector according to the environmental performance are shown in \autoref{fig:multi_industries_clustering_result_environemental_performance}.
Some industry sectors show a clear tendency towards a specific cluster, while others are less evenly distributed.
Therefore, general cluster structures can be identified with industry sectors being ranked by the chosen metric.
Also, companies within industry sectors can be compared, which helps detect outliers and best practices.

\begin{figure}
    \centering
    \includegraphics[width=0.8\textwidth]{figures/liu_assessmentOfIndustries/liu_environmentalPerformance.jpg}
    \caption{Multi-Industries Clustering Result based on the $SO_2$ Emission \cite{LIU-BDE}}
    \label{fig:multi_industries_clustering_result_environemental_performance}
\end{figure}

\begin{table}[h]
    \centering
    \begin{tabular}{c|c|c|c|c}
        \texttt{Fields} & \texttt{Sum} & \texttt{Cluster 0} & \texttt{Cluster 1} & \texttt{Cluster 2} \\
        \hline
        Chemical Industry & 144 & 138 & 0 & 6 \\
        Coal Mining and Washing Industry & 60 & 54 & 4 & 2 \\
        Textile Industry & 49 & 49 & 0 & 0 \\
        Paper Products Industry & 47 & 42 & 0 & 5 \\
    \end{tabular}
    \caption{Snippet of the Multi-Industries Clustering Results based on the $SO_2$ Emission \cite{LIU-BDE}}
    \label{tab:multi_industries_clustering_results_based_on_the_so2_emission}
\end{table}

\paragraph*{MAL-HBP:}
% Wie soll dieser Paragraph strukturiert sein?
% 1. Allgemeine Ergebnisse ansprechen
% 2. Auf ein Beispiel eingehen (in diesem Fall das Clustering und die Tabelle)


\begin{figure}
    \centering
    \includegraphics[width=0.8\textwidth]{figures/malatesta_hsop/malatesta_routinisedHousehold.jpg}
    \caption{K-Means resulting in Routinised Household Using all Year Energy Data \cite{MAL-HBP}}
    \label{fig:routinized_household}
\end{figure}

\begin{figure}
    \centering
    \includegraphics[width=0.8\textwidth]{figures/malatesta_hsop/malatesta_unroutinisedHousehold.jpg}
    \caption{K-Means resulting in Unroutinised Household Using all Year Energy Data \cite{MAL-HBP}}
    \label{fig:non_routinized_household}
\end{figure}

Variable lifestyles with different work commitments (e.g. working from home) or family structures (e.g. having children) alter the energy consumption of households \cite{KUR-HBP}.
If occupants are routinized, their behaviors are repetitive \cite{BRE-EWP}.
\section{Identifying and Differentiating Natural Disaster- and Electrical Fault-Impacted Load Profiles}
\label{sec:identifying_and_differentiating_natural_disaster_and_electrical_fault_impacted_load_profiles}
% One paragraph for each research question.
% Remember to include equations and figures, and cite everything!