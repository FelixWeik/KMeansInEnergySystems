\chapter{Findings}
\label{cha:findings}

% \begin{itemize}
%     \item Steps taken to arrive at the presented findings 
%     \item research question is addressed
%     \item present findings
%     \item graphs and table support main findings 
%     \item findings are presented without judgment
% \end{itemize}

% Wie möchte ich die Research Questions erklären?
% - Zusammenhängend => Mehr ein Fließtext, der die Fragen mithilfe der Paper beantwortet
% - Anhand der Ergebnisse des Papers eigene Rückschlüsse ziehen
% - Dabei immer im Kontext der Research Questions bleiben
% - Jedes Ergebnis soll so erklärt werden, dass es in den jeweiligen Kontext hineinpasst
% - Viele anschauliche Grafiken verwenden
% - Dabei alles sauber labeln und möglichst viel zitieren
\section{Improving the Design of Energy-Saving Incentives}
\label{sec:improving_the_design_of_energy_saving_incentives}
\subsection*{Structure Thougt 2}
Explaining the questions.
Separating the examples, yet keep the introduction as short as possible.
Presenting and explaining the findings.

\section{Identifying and Differentiating Natural Disaster- and Electrical Fault-Impacted Load Profiles}
\label{sec:identifying_and_differentiating_natural_disaster_and_electrical_fault_impacted_load_profiles}
One paragraph for each research question.
Remember to include equations and figures, and cite everything!