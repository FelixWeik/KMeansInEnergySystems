\chapter{Introduction}
\label{cha:introduction}

% \begin{itemize}
%     \item Relevance of thesis is explained (Check)
%     \item Give overall context (Check)
%     \item Gap in research is identified (Check)
%     \item Set research questions (Check)
%     \item The Goal of the thesis is formulated clearly (Check)
%     \item How will the problem be addressed (Check)
%     \item brief overview of the paper's structure (Check)
% \end{itemize}

% Explain the relevance of the thesis
Loads of data are created anywhere everywhere all the time.
Yet, due to the lack of actuators collecting and analyzing the data, the data is not fully utilized.
Without proper analysis, the data is just a bunch of numbers.
By analysing the data one can understand the creation process of our most prevalent problems, being able to tackle them at the root.
Without fully understanding the cause of a problem, one can only treat the symptoms.
% Gap in research is identified
A tool to generate knowledge and insights from data efficiently is needed.

% Give overall context
This review introduces the k-means clustering algorithm as a tool to analyse data, and to identify patterns and relations.
K-means is a simple yet efficient, unsupervised machine-learning algorithm that is used to cluster data.
This research is done in a certain context, namely the energy sector.

% Set research questions
After introducing the k-means algorithm and its corresponding initialization techniques, the algorithm is applied to the given context.
This happens by answering two focal research questions with the first question being "\textit{How does the k-means Algorithm help to understand Electricity Load Patterns?}".
Outgoing from this question, conclusions on the second research question are drawn.
The second research question is "\textit{How can the k-means Algorithm be applied to improve the Design of Energy Saving Incentives?}".

% Brief overview of the paper's structure + how will the problem be addressed
This work is structured into several chapters.
The main focus is on the literature review, finding answers to the given research questions.
First, the k-means algorithm and its corresponding concepts are introduced.
Also, the methodology of the research is explained.
Then, the findings of the papers on the research questions are summarized and discussed.
Outlooks on future improvements and follow-up questions are given.
Finally, \autoref{cha:conclusion} concludes the review.