\section{Introduction}
\label{cha:introduction}

% \begin{itemize}
%     \item Relevance of thesis is explained (Check)
%     \item Give overall context (Check)
%     \item Gap in research is identified (Check)
%     \item Set research questions (Check)
%     \item The Goal of the thesis is formulated clearly (Check)
%     \item How will the problem be addressed (Check)
%     \item brief overview of the paper's structure (Check)
% \end{itemize}

% Explain the relevance of the thesis
The number of climate change-impacted natural disasters is increasing \cite{JES-IND}.
Damages to the electrical infrastructure, power outages, and increasing maintenance costs are the consequences \cite{FAN-CCI}.
A key aspect in fighting an ever-accelerating climate change is the transition towards a sustainable energy system.

% Rising temperatures, natural disasters, extreme weather events and rising sea levels are just the beginning of the consequences we are facing in the future.
This transition is supported by understanding and analyzing data that is already available.
Due to the lack of actuators collecting and analyzing the data, datasets are not fully utilized \cite{LIU-BDE}.
% Gap in research is identified
To get a better understanding of the data, a tool to generate knowledge and insights from data efficiently is needed.

% Give overall context
This review introduces such a tool - the k-means clustering algorithm.
K-means is a simple, efficient, unsupervised machine-learning algorithm that is used to cluster data, identifying patterns and relations.
This research is done in a certain context, namely the energy sector.

% Set research questions
After introducing the k-means algorithm and its corresponding initialization techniques, the algorithm is applied to the given context.
This answers two research questions.
Question one finds answers on "\textit{How does the k-means algorithm help to understand electricity load patterns?}".
Starting from question one, initial conclusions can be drawn regarding the second research question: "\textit{How can the k-means algorithm be applied to improve the design of energy saving incentives?}".
% After introducing energy systems say something about the data
These questions are answered by reviewing different applications of the k-means algorithm in private housing, industry, and electricity load patterns of Lombok, Indonesia.

% Brief overview of the paper's structure + how will the problem be addressed
To find answers to the given research questions, this work is based on a literature review.
The structure of the review is as follows.
Chapter \ref{cha:background} introduces the k-means algorithm and its corresponding concepts.
Also, the methodology of the research is explained.
Chapter \ref{cha:findings} presents the findings of the reviewed papers and their impact on the research questions.
Then, Chapter \ref{cha:discussionAndConclusion} points out the limitations of the thesis, k-means' strengths and weaknesses, and a brief comparison of different clustering methods.
The results are discussed and the conclusion is drawn.
Finally, outlooks on future improvements are given.