\chapter{Introdcution}
\label{cha:introduction}

% \begin{itemize}
%     \item Relevance of thesis is explained
%     \item Give overall context
%     \item Gap in research is identified
%     \item Set research questions
%     \item The Goal of the thesis is formulated clearly
%     \item How will the problem will be addressed
%     \item brief overview of the paper's structure
% \end{itemize}

% Ansatz 1:
% Climate change is one of the most pressing issues of our time. 
% The effects of climate change are already being felt around the world, and the consequences are expected to be even more severe in the future.
% Therefore, everyone has to contribute to the reduction of greenhouse gas emissions.

% \begin{enumerate}
%     \item k-means identifies different energy usage patterns in an industrial context
%     \item k-means identifies different usage patterns and routines in homes
%     \item k-means identifies energy load profiles post and before natural disasters and electrical faults
%     \item k-means identifies the relation between refugees, weather conditions and natural disasters
% \end{enumerate}

% Ansatz 2:
Loads of data are created anywhere everywhere all the time.
Yet, the data is not fully utilized.
Without proper analysis, the data is just a bunch of numbers.
By analysing the data one can understand the creation process of our most prevalent problems, being able to tackle them at the root.
Without fully understanding the cause of a problem, one can only treat the symptoms.

This paper introduces the k-means clustering algorithm as a tool to analyse data, and to identify patterns and relations.
This is done in a certain context, namely the energy sector.
K-means is an unsupervised machine-learning algorithm that is used to cluster data.
It is a simple yet efficient algorithm that can be applied to large datasets.

After introducing the k-means algorithm and its corresponding data mining techniques, this algorithm dives into the given context.
The main part takes a look at two research questions:
\paragraph*{Improving the Design of Energy-Saving Incentives:} 
By closely analyzing the energy usage profiles of households and industry sectors, one can identify different usage patterns.
By identifying routines, patterns and relations in energy usage and the according values, one can design energy-saving incentives that are tailored to the needs of the consumer.
\paragraph*{Identifying and Differentiating Natural Disaster- and Electrical Fault-Impacted Load Profiles:}
K-means is applied to data representing the energy load profiles over several years.
This enables clustering of the data in pre- and post-disaster clusters.
By comparing the two clusters, one can identify the impact of natural disasters and electrical faults on the energy load profiles.
