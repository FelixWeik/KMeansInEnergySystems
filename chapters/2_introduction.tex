\section{Introduction}
\label{cha:introduction}

% \begin{itemize}
%     \item Relevance of thesis is explained (Check)
%     \item Give overall context (Check)
%     \item Gap in research is identified (Check)
%     \item Set research questions (Check)
%     \item The Goal of the thesis is formulated clearly (Check)
%     \item How will the problem be addressed (Check)
%     \item brief overview of the paper's structure (Check)
% \end{itemize}

% Explain the relevance of the thesis
The number of climate change-impacted natural disasters is increasing \cite{JES-IND}.
Damages to the electrical infrastructure, power outages, and increasing maintenance costs are the consequences \cite{FAN-CCI}.
A key aspect in fighting an ever-accelerating climate change is the transformation towards a sustainable energy system.

% Rising temperatures, natural disasters, extreme weather events and rising sea levels are just the beginning of the consequences we are facing in the future.
This transition is supported by understanding and analyzing data that is already available.
Due to the lack of actuators collecting and analyzing the data, it is not fully utilized \cite{LIU-BDE}.
% Gap in research is identified
To get a better understanding of the data a tool to generate knowledge and insights from data efficiently is needed.

% Give overall context
This review introduces such tool, namely the k-means clustering algorithm to analyze data, and to identify patterns and relations.
K-means is a simple, efficient, unsupervised machine-learning algorithm that is used to cluster data.
This research is done in a certain context, namely the energy sector.

% Set research questions
After introducing the k-means algorithm and its corresponding initialization techniques, the algorithm is applied to the given context.
This happens by answering two focal research questions with the first question being "\textit{How does the k-means Algorithm help to understand Electricity Load Patterns?}".
Outgoing from this question, conclusions on the second research question are drawn.
The second research question is "\textit{How can the k-means Algorithm be applied to improve the Design of Energy Saving Incentives?}".
% After introducing energy systems say something about the data
These questions are answered by reviewing different applications of the k-means algorithm in private housing, industry, and electricity load patterns of Lombok, Indonesia.

% Brief overview of the paper's structure + how will the problem be addressed
The main focus of this work is on the literature review, finding answers to the given research questions.
The structure of the review is as follows.
First, in Chapter \ref{cha:background} the k-means algorithm and its corresponding concepts are introduced.
Also, the methodology of the research is explained.
Chapter \ref{cha:findings} presents the findings of the reviewed papers and their impact on the research questions.
Then, Chapter \ref{cha:discussionAndConclusion} points out the limitations of the thesis, k-means strengths and weaknesses, and a brief comparison of clustering methods.
A discussion of the results and a conclusion are given.
Finally, outlooks on future improvements and follow-up questions are given.