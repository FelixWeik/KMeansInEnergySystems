\chapter{Introdcution}
\label{cha:introduction}

% \begin{itemize}
%     \item Relevance of thesis is explained (Check)
%     \item Give overall context (Check)
%     \item Gap in research is identified (Check)
%     \item Set research questions (Check)
%     \item The Goal of the thesis is formulated clearly (Check)
%     \item How will the problem be addressed (Check)
%     \item brief overview of the paper's structure (Check)
% \end{itemize}

% Explain the relevance of the thesis
Loads of data are created anywhere everywhere all the time.
Yet, due to the lack of actuators collecting and analyzing the data, the data is not fully utilized.
Without proper analysis, the data is just a bunch of numbers.
By analysing the data one can understand the creation process of our most prevalent problems, being able to tackle them at the root.
Without fully understanding the cause of a problem, one can only treat the symptoms.
% Gap in research is identified
A tool to generate knowledge and insights from data is needed.

% Give overall context
This review introduces the k-means clustering algorithm as a tool to analyse data, and to identify patterns and relations.
K-means is a simple yet efficient, unsupervised machine-learning algorithm that is used to cluster data.
This research is done in a certain context, namely the energy sector.

% Set research questions
After introducing the k-means algorithm and its corresponding data mining techniques, the algorithm is applied to the given context.
The following focal research questions reflect the purpose of this work:
\textit{How can k-means help to improve the design of energy-saving incentives?} and \textit{Can k-means identify and differentiate natural disaster-impacted load profiles?}.

% Brief overview of the paper's structure + how will the problem be addressed
This work is structured into several chapters, each section reviewing a single research question.
These sections focus on the literature review, all in the context of the research question.
First, \autoref{cha:background} introduces the background work on the k-means algorithm and the research questions.
Then, \autoref{sec:improving_the_design_of_energy_saving_incentives} reviews the literature on the first research question.
This review work will be presented from two perspectives: the perspective of energy efficiency in China's industry and the perspective of behavioral practices in private housing.
Third, \autoref{sec:identifying_and_differentiating_natural_disaster_and_electrical_fault_impacted_load_profiles} reviews the literature on the second research question, the identification and differentiation of natural disaster- and electrical fault-impacted load profiles. 
This review work will deep dive into the energy sector of Lombok, Indonesia.
By reviewing Lombok's climate resilience over several, natural disaster-impacted years, the application of k-means is illustrated.
\autoref{cha:discussion} outlines the current issues with recommended future improvements.
Finally, \autoref{cha:conclusion} concludes this review.