
\section{Outlook}
\label{sec:outlook}
% Auf climate change eingehen
The main goal of the thesis is to use k-means to design energy-saving incentives and understand energy load patterns.
Still, there are ways to improve the research and the results in the future.

% Connect to the limitations
As mentioned in the limitations of the thesis, the reviewed papers are \textit{biased towards their energy source}.
To fully understand energy load patterns and design universally applicable energy-saving incentives, the research needs to be extended.
This can be done by including different cultural and geographical regions, different environmental conditions and different energy sources over a larger timespan.
Additionally, this will eliminate the bias towards the COVID-19 pandemic due to the larger timespan.
This implies the need for a larger dataset to be analyzed in the future.

Also, the \textit{correctness of the k-means algorithm's results} needs to be proven in further research since it was not communicated in the reviewed papers.
The algorithm has to be executed multiple times to ensure the correctness of the results.
Another possibility is to use different clustering algorithms like the introduced k-harmonic means or MAP-DP, comparing the results and reviewing their correctness manually or by using data analysis tools.
To verify that no contrary position can be proven by the clustering results, one can assume the contrary position to be true and check whether the clustering results can be applied in a way that proves the contrary position wrong.

Future research should also focus on \textit{applying higher-level models to the clustering results} to generate more knowledge out of the data.
Like Malatesta et al. \cite{MAL-HBP} did, the clustering results are just the first step in making data more available and answering the research questions.
Further research must focus on applying higher-level models to the clustering results.

Finally, \textit{testing different values for $k$} is essential for further research.
Jessen et. al \cite{JES-IND} fail to spot single outliers without a general trend in the data.
Ongoing research focussing on testing different values for $k$ can help to identify these outliers since the algorithm might spot and assign them to a single cluster.

% What can people do in the future 
In the future, further research with larger, more different datasets and different clustering algorithms is needed to fully understand and answer the research questions.