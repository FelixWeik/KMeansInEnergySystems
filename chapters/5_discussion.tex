\chapter{Discussion \& Outlook}
\label{cha:discussion}

% discuss limitations of the thesis
% potential next steps

% Was möchte ich in diesem Abschnitt vermitteln?
% - Viele eigene Gedanken einbringen
% - Diskutieren, was eventuell besser gemacht werden kann
%   - Können bessere Ergebnisse mit anderen Methoden erzielt werden?
%   - Ist die Arbeit mit anderen Daten sinnvoll?
% - Hat die gegebene Recherche Einschränkungen?

% A case study with larger shares of renewable energy resources in the energy and capacity mix is recommended to investigate the impacts of natural disasters on the renewable-based electricity system [LIU-BDE]

\section{Discussion}
\subsection{Limitations of the Thesis}
% Wie kann man die Limitations der Paper und der Findings auf die zu beantwortenden Research Questions anwenden?
% Also welche Aspekte der Research Question weißen Lücken auf oder wurden unzureichend behandelt?

\subsection{Strengths and Limitations of K-Means}
% Grundlegende Idee: Auf Limitations und Findings die Stärken und Schwächen von K-Means ausformulieren
\begin{description}
    \item[Strengths:]
    \item[Weaknesses:]  
    % Not a good choice for high-dimensional data
\end{description}
\subsection{Comparison of Clustering Methods}
\begin{itemize}
    % Raykov
    \item MAP-DP \cite{HAM-ALT}
    % Hamerly
    \item k-harmonic means \cite{RAY-ALT}
    \item fuzzy k-means \cite{RAY-ALT}
    \item Gaussian expectation-maximization \cite{RAY-ALT}
    % Colombo
    \item K-medoids; this and the rest from \cite{COL-ALT}
    \item BIRCH
    \item Bayesian clustering
    \item HDBSCAN
    \item CLIQUE
    \item SPECTRAL
    \item SOMs
    \item TICC
\end{itemize}

\section{Outlook}
% Was kann in Zukunft besser gemacht werden?