\chapter{Discussion \& Outlook}
\label{cha:discussion}

% discuss limitations of the thesis
% potential next steps

% Was möchte ich in diesem Abschnitt vermitteln?
% - Viele eigene Gedanken einbringen
% - Diskutieren, was eventuell besser gemacht werden kann
%   - Können bessere Ergebnisse mit anderen Methoden erzielt werden?
%   - Ist die Arbeit mit anderen Daten sinnvoll?
% - Hat die gegebene Recherche Einschränkungen?

% A case study with larger shares of renewable energy resources in the energy and capacity mix is recommended to investigate the impacts of natural disasters on the renewable-based electricity system [LIU-BDE]

\section{Discussion}
%//TODO: Vergleich in den Strengths einbauen

\subsection{Strengths and Limitations of K-Means}
% Grundlegende Idee: Auf Limitations und Findings die Stärken und Schwächen von K-Means ausformulieren
\begin{description}
    \item[Strengths:]
    K-means is simple and easy to use on datasets.
    A lot of libraries and data analysis tools support the usage of k-means, for example, the r software environment.
    Jessen et al. \cite{JES-IND} showed that k-means is an effective tool for spotting general trends in large datasets.
    This also shows that k-means can easily be applied to large datasets like several thousand companies' environmental performance data in Liu et al. \cite{LIU-BDE}.
    With its linear runtime of $O(n * K * I)$ ($n$ being the number of data points, $k$ being the number of clusters, $i$ being the number of iterations), k-means is also fast in runtime.
    Finally, the algorithm converges every time, so a result without endless loops is guaranteed \cite{SEL-GCT}.
    % \begin{itemize}
    %     \item Simple and easy to implement and use since a wide range of libraries and tools are available (for example scikit-learn in Python)
    %     \item Effective in spotting general trends
    %     \item Converges every time
    %     \item Linear runtime of $O(n * K * I)$ with $n$ being the number of data points, $k$ being the number of clusters, $i$ being the number of iterations.
    % \end{itemize}
    \item[Weaknesses:]
    Jessen et al. \cite{JES-IND} showed that k-means sometimes failed to spot single outliers in the dataset.
    Outliers could only be spotted if they followed a general trend or if the dataset had multiple outliers in the same region that could be assigned to a single cluster.
    Since the algorithm converges to local optima, it has to be executed multiple times to ensure the correctness of the results.
    Furthermore, the initialization process is tedious since the number of clusters has to be provided before the algorithm can be executed:
    After finding the optimal number of clusters for the given problem, the initial cluster centers have to be found before being able to execute the algorithm on the given dataset.
    This makes the fast-executing algorithm slow in practice.
    % \begin{itemize}
    %     \item Failed to spot single outliers in the data
    %     \item Tideous initialization process until actual execution takes place
    %     \item Converges to local instead of global optima => needs to be executed multiple times
    %     \item Despite the algorithm being fast in runtime, running it multiple times and initializing/evaluating the results takes time
    % \end{itemize}
    %//TODO weniger yet
    %//TODO captions der bilder sollen nicht achsen beschreiben, sondern information first, da kann ruhig auch ein haufen stehen
\end{description}
\subsection{Comparison of Clustering Methods}
\begin{itemize}
    % Raykov
    \item MAP-DP \cite{HAM-ALT}
    % Hamerly
    \item k-harmonic means \cite{RAY-ALT}
    \item fuzzy k-means \cite{RAY-ALT}
    \item Gaussian expectation-maximization \cite{RAY-ALT}
    % Colombo
    \item K-medoids; this and the rest from \cite{COL-ALT}
    \item BIRCH
    \item Bayesian clustering
    \item HDBSCAN
    \item CLIQUE
    \item SPECTRAL
    \item SOMs
    \item TICC
\end{itemize}

\subsection{Limitations of the Thesis}
% Wie kann man die Limitations der Paper und der Findings auf die zu beantwortenden Research Questions und die Motivation anwenden?
% Also welche Aspekte der Research Question weißen Lücken auf oder wurden unzureichend behandelt?
The original motivation was to find and tackle the root causes of climate change in the energy sector by focusing on two factors: analyzing energy load profiles and energy-saving incentives.
The shown papers answered the given research questions to a certain extent.
However, some points were not fully addressed.

First, the papers were biased towards their choice of energy sources.
Lombok's energy load profiles portrayed by Jessen et al. \cite{JES-IND} stem mostly from fossil fuels.
The living lab in Blackwater, Australia, is powered sustainably by solar and using heat pumps \cite{MAL-HBP}.
Also, each paper focused on a single region with its unique cultural and environmental conditions.
Therefore the findings cannot be generalized or applied to other regions.
This concludes that the design of energy incentives cannot be applied universally by this research alone.
More research and test data using different energy sources and portraying different regions and environmental conditions are needed to fully prove the k-mean's effectiveness in designing energy-saving incentives. 

Additionally, it is not communicated to what extent the COVID-19 pandemic affected the research results.
Malatesta et al. \cite{MAL-HBP} found unroutinised patterns due to people working from home, which most likely is a result of the pandemic and people being forced to stay and work from home, adapting to this new situation.
This can distort the results and make them less applicable to normal conditions.
The pandemic also affects the total energy load profiles, which can lead to false clustering results.
Jessen et al. \cite{JES-IND} detected an increasing energy consumption over the years, especially in 2020 and 2021 but did not check whether these results stem from the pandemic or other factors.
Therefore, the results are not fully reliable and need to be re-evaluated under normal conditions.

Finally, the papers did not prove the correctness of the k-means algorithm's results.
Since the algorithm converges towards local optima, the results are not guaranteed to be the best possible solution.
Therefore, multiple executions are necessary to ensure the correctness of the results.
However, none of the papers performed multiple executions.
This can lead to result manipulation since an incorrect result can prove a wrong hypothesis despite the correct one being present.
From the research alone it is not possible to conclude that the k-means algorithm is the optimal choice for designing energy-saving incentives and understanding load patterns in general.


\section{Outlook}
\begin{itemize}
    \item Rerun the clustering algorithm for the same data with different values of k (maybe this helps spotting outliers)
    \item Check how local optima affect the clustering results (rerun with different initializations)
\end{itemize}
% Was kann in Zukunft besser gemacht werden?
%//TODO Connect to the limitations
%//TODO What can people do in the future 
%//TODO Auf climate change eingehen

%//TODO Save some space => template change if page limitations are exceeded