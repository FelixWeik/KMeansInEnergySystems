\chapter{Conclusion}
\label{cha:conclusion}

% summarize results, highlight main points
% cover results in the context of original motivation and problem statement
% "so what" message is formulated (key take homes, how does it help the reader in the future)
% discuss how the thesis addresses original questions

% Hauptaussage: K-Means ist toll um Trends zu spotten oder erste Eigenschaften aus einer Datenmenge herauszulesen, 
% allerdings kann man diesen Algorithmus nicht blind auf jedes Problem anwenden

% \subsection{Strengths and Weaknesses}
% The actual strengths and weaknesses of the algorithm will be discussed in \autoref{cha:conclusion}.

% \paragraph*{Strengths:}

% \paragraph*{Weaknesses:}
% Due to the algorithm's greedy nature, the algorithm may converge to a local minimum \cite{JAI-DCB}.
% Therefore, multiple runs for a given \texttt{k} value with different initial cluster centers are needed to obtain the optimal cluster result \cite{EZU-CPF}, \cite{BAR-LVG}.

The following conclusions were presented in the proseminar presentation:
Efficiency Asessment:
\begin{itemize}
    \item Revelation of hidden valuable knowledge (efficient analysis of big data masses; first step into making data more publicly available, allows to work with more advanced theories and models)
    \item industrial sectors, regions and companies can be compared (exemplary functions, targetted support)
\end{itemize}

HSOPs:
\begin{itemize}
    \item Social Theories can be linked to data (reveals even more knowledge; would not have been possible without clustering first)
    \item better development of best practices and habits (residents experience how their behavior affects the general load profiles)
    \item 
\end{itemize}