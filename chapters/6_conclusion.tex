\chapter{Conclusion}
\label{cha:conclusion}
%//TODO Retry to improve the pictures' resolution
% summarize results, highlight main points
% cover results in the context of original motivation and problem statement
% "so what" message is formulated (key take homes, how does it help the reader in the future)
% discuss how the thesis addresses original questions

% Hauptaussage: K-Means ist toll um Trends zu spotten oder erste Eigenschaften aus einer Datenmenge herauszulesen, 
% allerdings kann man diesen Algorithmus nicht blind auf jedes Problem anwenden

%//TODO: Auf Motivation eingehen
%//TODO: Eine "so what" message formulieren = How does the written report help the reader in the future?
%//TODO Include Key message and take homes
%//TODO Summarize results in three sentences

This review work introduced the k-means clustering algorithm as a tool to efficiently generate knowledge out of data.
This was done on two research questions, namely "\textit{How can k-means help to improve the design of energy-saving incentives?}" and "\textit{Can k-means identify and differentiate natural disaster-impacted load profiles?}".

Both research questions showed, that k-means is a powerful tool for spotting trends and characteristics in data.
It helps spot the most energy-intensive sectors and companies in big data masses, which helps provide targeted support and reduce information asymmetry \cite{LIU-BDE}.
It can lay the foundation for applying more advanced methods and theory, thus revealing even more knowledge \cite{MAL-HBP}.
Therefore, k-means is a powerful tool for finding trends, identifying characteristics and revealing hidden knowledge in big data masses.

Also, the flaws of the algorithm were discussed.
It was shown, that k-means cannot spot single outliers in big datasets \cite{JES-IND}.
Therefore, the detection of single earthquakes failed despite finding general load profile trends and seasonal patterns \cite{JES-IND}.
Furthermore, due to the algorithm being sensitive to initial starting conditions and the chosen value for $k$, multiple runs with different initial cluster centers are needed to obtain the optimal cluster result \cite{JAI-DCB}, \cite{EZU-CPF}, \cite{BAR-LVG}.
Also, result manipulation is possible due to the algorithm not always finding obvious cluster structures \cite{BOU-RPK}.
This makes the algorithm computationally expensive and time-consuming, despite the actual algorithm being simple and fast.

The choice of the best clustering algorithm should be guided by the given requirements of the data and the goals of the analysis \cite{COL-ALT}.
K-means clustering is among the most popular clustering algorithms.
This popularity is justified by its simplicity, efficiency in handling big datasets and effectiveness while delivering good results.
One must weigh the balance between finding general trends and patterns as suggested by k-means, and more explicit but less effective algorithms.
Consequently, k-means is a powerful tool and proves as the ideal algorithm for the given research.
