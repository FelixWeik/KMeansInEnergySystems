\chapter{Conclusion}
\label{cha:conclusion}

% summarize results, highlight main points
% cover results in the context of original motivation and problem statement
% "so what" message is formulated (key take homes, how does it help the reader in the future)
% discuss how the thesis addresses original questions

% Hauptaussage: K-Means ist toll um Trends zu spotten oder erste Eigenschaften aus einer Datenmenge herauszulesen, 
% allerdings kann man diesen Algorithmus nicht blind auf jedes Problem anwenden

% \subsection{Strengths and Weaknesses}
% The actual strengths and weaknesses of the algorithm will be discussed in \autoref{cha:conclusion}.

% \paragraph*{Strengths:}

% \paragraph*{Weaknesses:}
% Due to the algorithm's greedy nature, the algorithm may converge to a local minimum \cite{JAI-DCB}.
% Therefore, multiple runs for a given \texttt{k} value with different initial cluster centers are needed to obtain the optimal cluster result \cite{EZU-CPF}, \cite{BAR-LVG}.

