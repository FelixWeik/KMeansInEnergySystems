\chapter{Conclusion}
\label{cha:conclusion}
%//TODO Retry to improve the pictures' resolution
% summarize results, highlight main points
% cover results in the context of original motivation and problem statement
% "so what" message is formulated (key take homes, how does it help the reader in the future)
% discuss how the thesis addresses original questions

% Hauptaussage: K-Means ist toll um Trends zu spotten oder erste Eigenschaften aus einer Datenmenge herauszulesen, 
% allerdings kann man diesen Algorithmus nicht blind auf jedes Problem anwenden

%//TODO: Auf Motivation eingehen
%//TODO: Eine "so what" message formulieren = How does the written report help the reader in the future?
%//TODO Include Key message and take homes
%//TODO Summarize results in three sentences

This review work introduced the k-means clustering algorithm as a tool to efficiently generate knowledge out of data.
This was done on two research questions, namely "\textit{How can k-means help to improve the design of energy-saving incentives?}" and "\textit{Can k-means identify and differentiate natural disaster-impacted load profiles?}".

Both research questions showed, that k-means is a powerful tool for spotting trends and characteristics in data.
It helps spot the most energy-intensive sectors and companies in big data masses, which helps provide targeted support and reduce information asymmetry \cite{LIU-BDE}.
It can lay the foundation for applying more advanced methods and theory, thus revealing even more knowledge \cite{MAL-HBP}.
Therefore, k-means is a powerful tool for finding trends, identifying characteristics and revealing hidden knowledge in big data masses.

Also, the flaws of the algorithm were discussed.
It was shown, that k-means cannot spot single outliers in big datasets \cite{JES-IND}.
Therefore, the detection of single earthquakes failed despite finding general load profile trends and seasonal patterns \cite{JES-IND}.
Furthermore, due to the algorithm being sensitive to initial starting conditions and the chosen value for $k$, multiple runs with different initial cluster centers are needed to obtain the optimal cluster result \cite{JAI-DCB}, \cite{EZU-CPF}, \cite{BAR-LVG}.
Also, result manipulation is possible due to the algorithm not always finding obvious cluster structures \cite{BOU-RPK}.
This makes the algorithm computationally expensive and time-consuming, despite the actual algorithm being simple and fast.

The choice of the best clustering algorithm should be guided by the given requirements of the data and the goals of the analysis \cite{COL-ALT}.
K-means clustering is among the most popular clustering algorithms.
This popularity is justified by its simplicity, efficiency in handling big datasets and effectiveness while delivering good results.
One must weigh the balance between finding general trends and patterns as suggested by k-means, and more explicit but less effective algorithms.
Consequently, k-means is a powerful tool and proves as the ideal algorithm for the given research.

% The first research question was addressed from two different perspectives, private housing and the Chinese industry.
% Liu et al. \cite{LIU-BDE} used k-means to cluster industry sectors and individual companies based on chosen environmental and economic metrics.
% The clustering results helped identify the most energy-intensive sectors and companies, which can be used to target energy-saving incentives.
% Industrial sectors, regions and companies can be compared by their spotted performances, thus helping to provide targetted support.
% Also, information asymmetry between the government and the companies is reduced, as the disclosed data is fully analyzed despite the lack of actuators.
% Therefore, k-means is a powerful tool for finding characteristics and patterns in big data masses.
% Malatesta et al. \cite{MAL-HBP} used k-means to cluster private households based on their energy load profiles.
% Different HSOPs were identified, disclosing hidden knowledge about how best practices, habits and routines alter the standard energy consumption profile.
% Clustering results were then analyzed further by linking them to social theories and survey results.
% This helped to understand the underlying reasons for the different HSOPs.
% The results can be used to develop best practices and habits since the residents experience first hand on how their behaviors affect the general load profiles.
% This helps energy providers to better predict the energy demand and to plan accordingly.
% Also, the results can be used to design energy-saving incentives that are tailored to the different HSOPs.
% This showed, that k-means can lay the groundwork for applying more advanced theory and is therefore a powerful tool to preprocess data for further analysis.


% \subsection{Strengths and Weaknesses}
% The actual strengths and weaknesses of the algorithm will be discussed in \autoref{cha:conclusion}.

% \paragraph*{Strengths:}

% \paragraph*{Weaknesses:}
% Due to the algorithm's greedy nature, the algorithm may converge to a local minimum \cite{JAI-DCB}.
% Therefore, multiple runs for a given \texttt{k} value with different initial cluster centers are needed to obtain the optimal cluster result \cite{EZU-CPF}, \cite{BAR-LVG}.

% - Anhand der Ergebnisse des Papers eigene Rückschlüsse ziehen, dabei Grafiken und Tabellen aus den Papern verwenden
% - Dabei immer im Kontext der Research Questions bleiben
% - Hier werden also die Fragen beantwortet
% General Findings:
% \begin{itemize}
%     \item K-Means can be applied in different contexts by interpreting the results differently (Liu => industry assessment; Malatesta => Variable k with each cluster being a HSOP)
% \end{itemize}

% The following conclusions were presented in the proseminar presentation:
% Efficiency Assessment:
% \begin{itemize}
%     \item Revelation of hidden valuable knowledge (efficient analysis of big data masses; first step into making data more publicly available, allows to work with more advanced theories and models)
%     \item industrial sectors, regions and companies can be compared (exemplary functions, targetted support)
%     \item No more information asymmetry (disclosed data is fully analyzed, no more hidden information)
% \end{itemize}

% HSOPs:
% \begin{itemize}
%     \item Social Theories can be linked to data (reveals even more knowledge; would not have been possible without clustering first)
%     \item better development of best practices and habits (residents experience how their behavior affects the general load profiles)
%     \item 
% \end{itemize}

% Working from home is not observed with typical working structures and therefore not applicable to the standard energy load profile \cite{HAM-EED}. 