\section{Conclusion}
\label{cha:conclusion}
%//TODO conclusion mit discussion mergen, nicht mehr uterteilen
%//TODO Outlook am Schluss
% summarize results, highlight main points
% cover results in the context of original motivation and problem statement
% "so what" message is formulated (key take homes, how does it help the reader in the future)
% discuss how the thesis addresses original questions

% Hauptaussage: K-Means ist toll um Trends zu spotten oder erste Eigenschaften aus einer Datenmenge herauszulesen, 
% allerdings kann man diesen Algorithmus nicht blind auf jedes Problem anwenden






Also, the flaws of the algorithm were discussed.
It was shown, that k-means cannot spot single outliers in big datasets \cite{JES-IND}.
Therefore, the detection of single earthquakes failed despite finding general load profile trends and seasonal patterns \cite{JES-IND}.
Furthermore, due to the algorithm being sensitive to initial starting conditions and the chosen value for $k$, multiple runs with different initial cluster centers are needed to obtain the optimal cluster result \cite{JAI-DCB}, \cite{EZU-CPF}, \cite{BAR-LVG}.
Also, result manipulation is possible due to the algorithm not always finding obvious cluster structures \cite{BOU-RPK}.
This makes the algorithm computationally expensive and time-consuming, despite the actual algorithm being simple and fast.

% Summarize results in three sentences
% Include Key message and take homes
Through this, k-means makes data more publicly available and understandable, which is a key factor for efficient design and decision processes in creating energy-saving incentives.
Due to its simplicity, it can be applied to a wide range of problems, finding general trends and patterns.
However, it is not a one-size-fits-all solution and should be used with caution, as it can be computationally expensive and time-consuming.

