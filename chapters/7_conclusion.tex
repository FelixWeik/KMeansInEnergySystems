\section{Conclusion}
\label{cha:conclusion}
% summarize results, highlight main points
% cover results in the context of original motivation and problem statement
% "so what" message is formulated (key take homes, how does it help the reader in the future)
% discuss how the thesis addresses original questions

% Hauptaussage: K-Means ist toll um Trends zu spotten oder erste Eigenschaften aus einer Datenmenge herauszulesen, 
% allerdings kann man diesen Algorithmus nicht blind auf jedes Problem anwenden


% Auf Motivation eingehen
This review aimed to tackle climate change's root causes in the energy sector.
This meant understanding general electricity load patterns and using this knowledge to improve the design of energy-saving incentives.
K-means clustering was introduced as a tool for tackling these problems by applying it to different datasets in the context of two research questions.
The first research questions aimed to generally understand electricity load patterns by identifying patterns and characteristics in a given dataset.
Question two aimed to design energy-saving incentives in the context of private housing and industry.

Both research questions showed, that k-means is a powerful tool for spotting trends and characteristics in data.
It helps to identify the most energy-intensive sectors and companies in big data masses, which helps provide targeted support and reduce information asymmetry \cite{LIU-BDE}.
It identifies different consumer routines and habits in the consumption data which can lay the foundation for applying more advanced methods and theory, thus revealing even more knowledge \cite{MAL-HBP}.
Therefore, k-means is a powerful tool for finding trends, identifying characteristics and revealing hidden knowledge in big data masses.

Also, the flaws of the algorithm were discussed.
It was shown, that k-means cannot spot single outliers in big datasets \cite{JES-IND}.
Therefore, the detection of single earthquakes failed despite finding general load profile trends and seasonal patterns \cite{JES-IND}.
Furthermore, due to the algorithm being sensitive to initial starting conditions and the chosen value for $k$, multiple runs with different initial cluster centers are needed to obtain the optimal cluster result \cite{JAI-DCB}, \cite{EZU-CPF}, \cite{BAR-LVG}.
Also, result manipulation is possible due to the algorithm not always finding obvious cluster structures \cite{BOU-RPK}.
This makes the algorithm computationally expensive and time-consuming, despite the actual algorithm being simple and fast.

% Summarize results in three sentences
% Include Key message and take homes
Through this, k-means makes data more publicly available and understandable, which is a key factor for efficient design and decision processes in creating energy-saving incentives.
Due to its simplicity, it can be applied to a wide range of problems, finding general trends and patterns.
However, it is not a one-size-fits-all solution and should be used with caution, as it can be computationally expensive and time-consuming.

The choice of the best clustering algorithm should be guided by the given requirements of the data and the goals of the analysis \cite{COL-ALT}.
K-means clustering is among the most popular clustering algorithms.
This popularity is justified by its simplicity, efficiency in handling big datasets and effectiveness while delivering good results.
One must weigh the balance between efficiently identifying general trends and patterns as suggested by k-means, and more explicit but less effective algorithms.
%//TODO: Eine "so what" message formulieren = How does the written report help the reader in the future?
So, k-means proved as the ideal tool for the given research questions, it is a great starting point in understanding general electricity load patterns and designing energy-saving incentives.
