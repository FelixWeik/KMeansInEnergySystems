\chapter{Background}
\label{cha:background}
% existing methods and concepts are fully introduced
% Mathematics of the methods is correctly covered
% self-developed/chosen solution method is motivated and described fully

% Self-developed / chosen solution method is motivated and described fully
Summarize what the paper aims to do.
How will the following methodological concepts help to achieve the goal of the paper?
Why exactly are these concepts chosen?
What keywords were used to find the papers?
What were the search results?
How will the papers help in answering the research questions?


% Was haben sie in den Papers der Research Questions benutzt?


% Viel aus dem Paper über die Initialisierungsmethoden zitieren, da stehen viele tolle paper drin

% Was möchte ich in diesem Abschnitt vermitteln?
% - Vorstellung von K-Means:
%   - Vorstellung eines generellen Problems
%   - Funktionsweiße mit mathematischen Grundlagen
%   - Die Elbow-Methode
%   - Initialisierungsmethoden
%   - Stärken und Schwächen
%

\section{K-Means Clustering}
\label{sec:k_means_clustering}
% General definition/idea 
Organizing unlabeled data \cite{EZU-CPF} is the general problem of the clustering analysis.
The meaningful grouping of such unlabeled data is regarded as data clustering \cite{ABI-RKC}.
This allows for the identification of patterns and trends in the data, which can be used for further analysis, such as applying more advanced theories and methods.

% Introduce the general problem the algorithm tries to solve
K-means is a partitional clustering algorithm \cite{SIN-UKC}.
It is used to group a set of n data points from z dimensions into k clusters.
This means, that a single partition of the initial dataset is produced with each point being assigned to a distinct cluster \cite{SIN-UKC}.
Clusters are produced heuristically while optimizing a criterion function defined globally on all data objects or locally on the subset of the data objects \cite{ZHU-EPC}.

\subsection{Mathematical Concepts}
As mentioned before, k-means divides a set of n data points from z dimensions into k clusters.
This is done by minimizing the sum of squared distances between a datapoint and its assigned cluster center within the whole cluster \cite{HAR-KMA}.
Doing this globally is an NP-hard problem.
Therefore, the algorithm seeks local optima, such that no point can be assigned to a different cluster and the result converges \cite{SEL-GCT}, \cite{HAR-KMA}.

The initial cluster centers are predefined as explained in \autoref*{subsec:initialisation_methods}.
Next, the means of the initial clusters are calculated.
Then, each data point is assigned to the cluster with the closest mean.
These steps are repeated until the cluster centers do not change anymore or the square sum of errors stays the same for multiple iterations \cite{HAR-KMA}.

\begin{enumerate}
    \item The algorithm requires an input matrix of n data points in z dimensions and the initial cluster centers as k points in z dimensions \cite{HAR-KMA}.
          The initial cluster centers are chosen according to the used initialization method as explained in \autoref*{subsec:initialisation_methods}.
    \item The average of each cluster is calculated by using $C_i = (1/M) \sum_{j=1}^{M}x_j$ with $C_i$ being the average of cluster $i$, $M$ the number of points in cluster $i$, and $x_j$ the $j$-th point in cluster $i$ \cite{SYA-IKC}.
    \item Iterate over all data points assigning each point to the nearest cluster center.
          To calculate the distance, the euclidean distance $d = \sqrt{(x_1-x_2)^2+(y_1-y_2)^2}$ are used.
    \item Steps 2 and 3 are repeated until the criterion function converges.
          The criterion function is $E=\sum_{i=1}^{k} \sum_{P \in C_i}|p-m_i|$ with $E$ being the square error-sum, $p$ the point in space, and $m_i$ the average of cluster $C_i$ \cite{LIU-BDE}.
\end{enumerate}
% Runtime complexity

\subsection{Finding k: The Elbow Method}
\subsection{Initialisation Methods}
\label{subsec:initialisation_methods}
\paragraph{RANDOM Approach}
\paragraph{Forgy Approach}
\subsection{Strengths and Weaknesses}
\paragraph*{Strenghts:}

\paragraph*{Weaknesses:}
Due to the algorithm's greedy nature, the algorithm may converge to a local minimum \cite{JAI-DCB}.
Therefore, multiple runs for a given \texttt{k} value with different initial cluster centers are needed to obtain the optimal cluster result \cite{EZU-CPF}, \cite{BAR-LVG}.

% Was sollen die folgenden beiden sections vermitteln?
%  Wichtig: Vergleichen zwischen verschiedenen Methoden, aber nur bei großen Unterschieden.
%  - Grundlegende Vorstellung der Methoden der Papers
%   - Aus welcher Perspektive gehen die Paper an die jeweilige Fragestellung heran?
%   - Warum verwendet das jeweilige Paper diese Methoden (wie sinnvoll ist das im gegebenen Kontext)? 
%   - Was möchten die jeweiligen Paper erreichen?
%   - In welchem Kontext stehen die Paper (bezogen auf die research questions)?
%   - Wie wenden die Paper k-means an?
%   - Dabei nicht jede Methodik einzeln aufschlüsseln, eher zusammenfassen.
%   - Werden die Daten danach weiter verarbeitet (weiterführende Theorien, ...)?
\section{Methodology of the Research Questions}
% Der Hauptfokus mag auf den drei Papern liegen, dabei aber möglichst viel anderes zitieren => Richtig böse die .bib vollklatschen
\subsection{K-Means in the Context of Energy-Saving Incentives}

\subsection{K-Means in the Context of Natural Disaster- and Electrical Fault-Impacted Load Profiles}
