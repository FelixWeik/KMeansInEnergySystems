\chapter{Background}
\label{cha:background}
% \begin{itemize}
%     \item existing methods and concepts are fully introduced
%     \item Mathematics of the methods is correctly covered
%     \item self-developed/chosen solution method is motivated and described fully
% \end{itemize}
% Self-developed / chosen solution method is motivated and described fully
Summarize what the paper aims to do.
How will the following methodological concepts help to achieve the goal of the paper?
Why exactly are these concepts chosen?
What keywords were used to find the papers?
What were the search results?
How will the papers help in answering the research questions?


% Was haben sie in den Papers der Research Questions benutzt?


% Viel aus dem Paper über die Initialisierungsmethoden zitieren, da stehen viele tolle paper drin

% Was möchte ich in diesem Abschnitt vermitteln?
% - Vorstellung von K-Means:
%   - Vorstellung eines generellen Problems
%   - Funktionsweiße mit mathematischen Grundlagen
%   - Die Elbow-Methode
%   - Initialisierungsmethoden
%   - Stärken und Schwächen
%

\section{K-Means Clustering}
\label{sec:k_means_clustering}
\subsection{A General k-means Problem}
\subsection{Mathematical Concepts}
\subsection{Finding k: The Elbow Method}
\subsection{Initialisation Methods}
\paragraph{RANDOM Approach}
\paragraph{Forgy Approach}
\subsection{Strengths and Weaknesses}


% Was sollen die folgenden beiden sections vermitteln?
%  Wichtig: Vergleichen zwischen verschiedenen Methoden, aber nur bei großen Unterschieden.
%  - Grundlegende Vorstellung der Methoden der Papers
%   - Aus welcher Perspektive gehen die Paper an die jeweilige Fragestellung heran?
%   - Warum verwendet das jeweilige Paper diese Methoden (wie sinnvoll ist das im gegebenen Kontext)? 
%   - Was möchten die jeweiligen Paper erreichen?
%   - In welchem Kontext stehen die Paper (bezogen auf die research questions)?
%   - Wie wenden die Paper k-means an?
%   - Dabei nicht jede Methodik einzeln aufschlüsseln, eher zusammenfassen.
%   - Werden die Daten danach weiter verarbeitet (weiterführende Theorien, ...)?
\section{Methodology of the Research Questions}
\subsection{K-Means in the Context of Energy-Saving Incentives}

\subsection{K-Means in the Context of Natural Disaster- and Electrical Fault-Impacted Load Profiles}
