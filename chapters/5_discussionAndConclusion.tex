% discuss limitations of the thesis
% potential next steps

% Was möchte ich in diesem Abschnitt vermitteln?
% - Viele eigene Gedanken einbringen
% - Diskutieren, was eventuell besser gemacht werden kann
%   - Können bessere Ergebnisse mit anderen Methoden erzielt werden?
%   - Ist die Arbeit mit anderen Daten sinnvoll?
% - Hat die gegebene Recherche Einschränkungen?

% A case study with larger shares of renewable energy resources in the energy and capacity mix is recommended to investigate the impacts of natural disasters on the renewable-based electricity system [LIU-BDE]

\section{Discussion and Conclusion}
\label{cha:discussionAndConclusion}

% Auf Motivation eingehen
This review takes the first step in the transition to more sustainable energy systems and infrastructures.
This means understanding general electricity load patterns and using this knowledge to improve the design of energy-saving incentives.
K-means clustering offers these functionalities.
This work applies it to different datasets in the context of two research questions.
The first research question aims to understand electricity load profiles by identifying patterns and characteristics in a given dataset.
Question two aims to design energy-saving incentives in the context of private housing and industry.

K-means helps to identify the most energy-intensive sectors and companies in big data masses, which helps provide targeted support and reduce information asymmetry \cite{LIU-BDE}.
Furthermore, k-means identifies different consumer routines and habits in the consumption data which can lay the foundation for applying more advanced methods and theory, thus revealing even more knowledge \cite{MAL-HBP}.
Therefore, k-means is a powerful tool for finding trends, identifying characteristics and revealing hidden knowledge in big data masses.

% \begin{itemize}
    %     \item Simple and easy to implement and use since a wide range of libraries and tools are available (for example scikit-learn in Python), Tideous initialization process until actual execution takes place
    %     \item Effective in spotting general trends, however Failed to spot single outliers in the data
    %     \item Converges every time, however Converges to local instead of global optima => needs to be executed multiple times
    %     \item Linear runtime of $O(n * K * I)$ with $n$ being the number of data points, $k$ being the number of clusters, $i$ being the number of iterations.
    %           Despite the algorithm being fast in runtime, running it multiple times and initializing/evaluating the results takes time
% \end{itemize}
The application of k-means in diverse contexts underscores its strengths and weaknesses.
The strengths of k-means lie in its simplicity and efficiency in handling big datasets.
This implies that k-means can easily take on large datasets like several thousand companies' environmental performance data \cite{LIU-BDE}.
Multiple libraries and data analysis tools support the usage of k-means, for example, the \texttt{R Software Environment} \cite{R-SOF}.
The initialization processes of finding the optimal number for $k$ and the initial cluster centers are taken care of by these libraries and tools.
However, these additional steps make the initialization process more time-consuming.
Jessen et al. \cite{JES-IND} show the effectiveness of k-means in spotting general trends in large datasets.
However, outliers can only be spotted if they follow a general trend or if the dataset has multiple outliers in the same region that can be assigned to a single cluster.
This limits the scope of k-means.
Finally, the algorithm converges every time, so a result is guaranteed \cite{SEL-GCT}.
Since it converges to local optima, multiple executions are necessary to ensure the correctness of the results.
This makes the fast-executing algorithm with its linear runtime of $O(n * k * i)$ ($n$ being the number of data points, $k$ being the number of clusters, $i$ being the number of iterations) more time-consuming and computationally expensive in practice.

% Vergleich zu anderen Algorithmen 
Different clustering algorithms are proven to improve the mentioned flaws.
K-harmonic means provides superior clustering results in low dimensions compared to k-means \cite{HAM-ALT}.
MAP-DP (Maximum a Posteriori Dirichlet Process) estimates the number of clusters from the data and therefore does not need the number of clusters to be provided beforehand \cite{RAY-ALT}.
Despite being less effective in runtime, MAP-DP is proven to deliver overall better results than k-means \cite{RAY-ALT}.

% Wie kann man die Limitations der Paper und der Findings auf die zu beantwortenden Research Questions und die Motivation anwenden?
% Also welche Aspekte der Research Question weißen Lücken auf oder wurden unzureichend behandelt?
The reviewed papers show some limitations and gaps in their research.
First, the papers are biased toward their choice of energy sources.
Lombok's energy load profiles by Jessen et al. \cite{JES-IND} stem mostly from fossil fuels.
The living lab in Blackwater, Australia, uses solar and heat pumps as sustainable power sources \cite{MAL-HBP}.
Also, each paper focuses on a single region with its unique cultural and environmental conditions.
Therefore the findings cannot be generalized or applied to other regions.
This concludes that the design of energy incentives cannot be applied universally by this research alone.
More research and test data using different energy sources and portraying different regions and environmental conditions are needed to fully prove the k-mean's effectiveness in designing energy-saving incentives. 
Also, in terms of sustainability, more research on sustainable energy sources is needed to generate knowledge regarding these energy sources.

Additionally, it is not communicated to what extent the COVID-19 pandemic affects the research results.
Malatesta et al. \cite{MAL-HBP} find non-routinized patterns due to people working from home, which most likely is a result of the pandemic and people being forced to stay and work from home, adapting to this new situation.
This distorts the results and makes them less applicable to normal conditions.
The pandemic also affects the total energy load profiles, which can lead to false clustering results.
Jessen et al. \cite{JES-IND} detect an increasing energy consumption over the years, especially in 2020 and 2021 but did not check whether these results stem from the pandemic or other factors.
Therefore, the results are not fully reliable and need to be re-evaluated under normal circumstances.

Finally, the papers do not prove the correctness of the k-means algorithm's results.
Since the algorithm converges towards local optima \cite{SEL-GCT}, the results are not guaranteed to be the best possible solution.
Therefore, multiple executions are necessary to ensure the correctness of the results.
However, none of the papers mention multiple executions on the same dataset.
This can lead to result manipulation since an incorrect result can prove a wrong hypothesis despite the correct one being present.

To conclude, the choice of the best clustering algorithm should be guided by the given requirements of the data and the goals of the analysis \cite{COL-ALT}.
K-means clustering is among the most popular clustering algorithms.
This popularity is due to its simplicity, efficiency in handling big datasets and effectiveness while delivering good results.
One must weigh the balance between efficiently identifying general trends and patterns as suggested by k-means, and more explicit but less effective algorithms.
% "So what" message = How does the written report help the reader in the future?
So, k-means proves as the ideal tool for the given research questions, it is a great starting point in understanding general electricity load patterns and designing energy-saving incentives.

