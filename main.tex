\documentclass{revtex4-2}
\bibliographystyle{plain}

% Preamble (add packages and custom commands here)
\usepackage{lipsum} % For generating dummy text
\usepackage{graphicx} % For including figures
\usepackage{float} % For controlling figure placement

% Title section
\begin{document}

\title{K-Means Clustering in Energy Systems}
\author{Felix Weik}
\date{\today}

% Usually an abstract includes the following.
% A brief introduction to the topic that you're investigating. (passt)
% Explanation of why the topic is important in your field/s. (passt)
% Your research question/s / aim/s. (sind drin)
% An indication of your research methods and approach. (auch)
% Your key message.
% A summary of your key findings.
% An explanation of why your findings and key message contribute to 

% Abstract section
\begin{abstract}
Climate change is one of the most pressing issues of our time. 
Its impacts are already visible and will continue to grow in the future.
In the energy sector too, climate change can already be observed, be it the electrical infrastructure collapse due to the Ahrtal flood in 2021 or the Californian wildfires in 2020.
Also, due to increasing climate awareness, people are looking for ways to optimize their energy usage reducing their carbon footprint.

Therefore, this work focuses on a tool to understand energy load patterns and thus improve the design of energy-saving incentives.
This tool is the k-means clustering algorithm.
K-means clustering is one of the most popular and widely used clustering algorithms due to its simplicity and efficiency in handling large datasets.
This work reviews its applications to understand energy load patterns and thus improve the design of energy-saving incentives.
The reviewed papers demonstrate k-means' applications from different perspectives, such as identifying home systems of best practices or reviewing energy efficiency metrics of Chinese industry sectors.

The results show that k-means works well in finding trends and patterns while handling large datasets efficiently.
This generates knowledge out of the data, which can be used as input for higher-level models or decision-making processes.
Also, k-means weaknesses in this area of usage are analyzed, namely the need for a priori knowledge for the value of $k$ or the missing ability to spot single outliers in the data.
The algorithm is efficient in its computational complexity, however, multiple executions are needed to evade the risk of local optima.
This contributes to the understanding of energy load patterns and thus improves the design of energy-saving incentives.

The research shows that k-means can indeed be used to understand energy load patterns and to improve the design of energy-saving incentives.
\end{abstract}

\maketitle

% Introduction section
\chapter{Introdcution}
\label{cha:introduction}

% \begin{itemize}
%     \item Relevance of thesis is explained (Check)
%     \item Give overall context (Check)
%     \item Gap in research is identified (Check)
%     \item Set research questions (Check)
%     \item The Goal of the thesis is formulated clearly (Check)
%     \item How will the problem be addressed (Check)
%     \item brief overview of the paper's structure (Check)
% \end{itemize}

% Explain the relevance of the thesis
Loads of data are created anywhere everywhere all the time.
Yet, due to the lack of actuators collecting and analyzing the data, the data is not fully utilized.
Without proper analysis, the data is just a bunch of numbers.
By analysing the data one can understand the creation process of our most prevalent problems, being able to tackle them at the root.
Without fully understanding the cause of a problem, one can only treat the symptoms.
% Gap in research is identified
A tool to generate knowledge and insights from data is needed.

% Give overall context
This review introduces the k-means clustering algorithm as a tool to analyse data, and to identify patterns and relations.
K-means is a simple yet efficient, unsupervised machine-learning algorithm that is used to cluster data.
This research is done in a certain context, namely the energy sector.

% Set research questions
After introducing the k-means algorithm and its corresponding data mining techniques, the algorithm is applied to the given context.
The following focal research questions reflect the purpose of this work:
\textit{How can k-means help to improve the design of energy-saving incentives?} and \textit{Can k-means identify and differentiate natural disaster-impacted load profiles?}.

% Brief overview of the paper's structure + how will the problem be addressed
This work is structured into several chapters, each section reviewing a single research question.
These sections focus on the literature review, all in the context of the research question.
First, \autoref{cha:background} introduces the background work on the k-means algorithm and the research questions.
Then, \autoref{sec:improving_the_design_of_energy_saving_incentives} reviews the literature on the first research question.
This review work will be presented from two perspectives: the perspective of energy efficiency in China's industry and the perspective of behavioral practices in private housing.
Third, \autoref{sec:identifying_and_differentiating_natural_disaster_and_electrical_fault_impacted_load_profiles} reviews the literature on the second research question, the identification and differentiation of natural disaster- and electrical fault-impacted load profiles. 
This review work will deep dive into the energy sector of Lombok, Indonesia.
By reviewing Lombok's climate resilience over several, natural disaster-impacted years, the application of k-means is illustrated.
\autoref{cha:discussion} outlines the current issues with recommended future improvements.
Finally, \autoref{cha:conclusion} concludes this review.

% Methods section
\section{Background}
\label{cha:background}
% existing methods and concepts are fully introduced
% Mathematics of the methods is correctly covered
% self-developed/chosen solution method is motivated and described fully

% Self-developed / chosen solution method is motivated and described fully
% Summarize what the paper aims to do.
% How will the following methodological concepts help to achieve the goal of the paper?
% Why exactly are these concepts chosen?
% What keywords were used to find the papers?
% What were the search results?
% How will the papers help in answering the research questions?


% Was haben sie in den Papers der Research Questions benutzt?


% Viel aus dem Paper über die Initialisierungsmethoden zitieren, da stehen viele tolle paper drin

% Was möchte ich in diesem Abschnitt vermitteln?
% - Vorstellung von K-Means:
%   - Vorstellung eines generellen Problems
%   - Funktionsweiße mit mathematischen Grundlagen
%   - Die Elbow-Methode
%   - Initialisierungsmethoden
%   - Stärken und Schwächen

\subsection{K-Means Clustering}
\label{sec:k_means_clustering}
% General definition/idea 
Organizing unlabeled data \cite{EZU-CPF} is the general problem of the clustering analysis.
The meaningful grouping of such unlabeled data is regarded as data clustering \cite{ABI-RKC}.
This allows for the identification of patterns and trends in the data, which can be used for further analysis, such as applying more advanced theories and methods.

% Introduce the general problem the algorithm tries to solve
K-means is a partitional clustering algorithm \cite{SIN-UKC}.
It is used to group a set of n data points from z dimensions into k clusters.
This means, that a single partition of the initial dataset is produced with each point being assigned to a distinct cluster \cite{SIN-UKC}.
Clusters are produced heuristically while optimizing a criterion function defined globally on all data objects or locally on the subset of the data objects \cite{ZHU-EPC}.

\subsubsection{Mathematical Concepts}
K-Means clustering utilizes the concept of minimizing the sum of squared distances between a datapoint and its assigned cluster center within the whole cluster \cite{HAR-KMA}
Doing this globally exists, but it is computationally expensive \cite{LIS-GKC}.
Therefore, the algorithm seeks local optima, such that no point can be assigned to a different cluster and the result converges \cite{SEL-GCT}.
An execution of the general algorithm is shown in \ref{subsec:general_algorithm}.

\subsubsection{Initialisation Methods}
\label{subsec:initialisation_methods}
Initialization methods are used to find the initial cluster centers and the initial clustering.
They provide the first step in the k-means algorithm.
There are multiple methods to do so, with the most popular approaches being \texttt{RANDOM} \cite{PEN-ECI}, the \texttt{Forgy Approach} \cite{AND-CAA}, the \texttt{Macqueen Approach} \cite{MCQ-MCA}, and the \texttt{Kaufman Approach} \cite{KAU-FGD}.

The \texttt{RANDOM} approach is the most commonly used method.
It is provided with the datasets and a value for $k$ and then randomly selects $k$ points from the dataset as the initial cluster centers.
Iterating over the dataset, each point is assigned to a random cluster center.
\texttt{RANDOM} is very popular due to it being an effective initialization method in terms of convergence speed and robustness while being fast in execution \cite{PEN-ECI}.
Therefore it is assumed to be the default initialization method in this thesis.

Other initialization methods differ in the way they choose the initial cluster centers or assign the initial clusters.
\texttt{FORGY} provides a similar approach to \texttt{RANDOM}, but the points are initially assigned to the nearest randomly chosen cluster center.
The usual distance metric is the Euclidean distance \cite{AND-CAA}.
Despite this being a seemingly better first step, the \texttt{RANDOM} approach has similar results in practice in terms of convergence speed while being less computationally expensive \cite{AND-CAA}.

\subsubsection{General Algorithm}
\label{subsec:general_algorithm}
K-means can be implemented in multiple variants.
The most general implementation using the Euclidean distance is shown in the following.
\begin{enumerate}
    \item The algorithm requires an input matrix of n data points in z dimensions and the initial cluster centers as k points in z dimensions \cite{HAR-KMA}.
          The initial cluster centers and the initial clustering are chosen according to the used initialization method as explained in \ref{subsec:initialisation_methods}.
          This review assumes the \texttt{RANDOM} approach as the default initialization method.
    \item The average of each cluster is calculated by using $C_i = \frac{1}{M} \sum_{j=1}^{M}x_j$ with $C_i$ being the average of cluster $i$, $M$ the number of points in cluster $i$, and $x_j$ the $j$-th point in cluster $i$ \cite{SYA-IKC}.
          $C_i$ is then used as the new cluster center of cluster $i$.
    \item Iterate over all data points assigning each point to the nearest cluster center.
          To calculate the distance, the euclidean distance $d = \sqrt{(x_1-x_2)^2+(y_1-y_2)^2}$ is used.
    \item Steps 2 and 3 are repeated until the criterion function converges.
          The criterion function (Equation \ref{eq:sse}) represents the sum of square errors.
          It is defined as \begin{equation}\label{eq:sse}E=\sum_{i=1}^{k} \sum_{P \in C_i}|p-m_i|\end{equation} with $p$ being the point in space, and $m_i$ being the average of cluster $C_i$ \cite{LIU-BDE}.
\end{enumerate}

\subsubsection{Finding k: The Elbow Method}
The elbow method is a heuristic to find the optimal number $k$ of clusters for a given problem \cite{SYA-IKC}.
It is calculated by plotting the square sum of errors (\ref{eq:sse}) against the number of clusters.
The optimal number of clusters is the point where the graph's slope changes the most, the so-called elbow \cite{SYA-IKC}.

\subsubsection{Data Scaling: The MinMaxScaler}
The MinMaxScaler is a data scaling tool of the \texttt{Scikit Learn} library \cite{SKL-MMS}.
It scales given data to a range of 0 to 1 and keeps the shape of the original distribution.
Most of the following papers use this scaler before applying k-means clustering.
For each value $x$ in the dataset is scaled applying \begin{equation}\label{eq:minmaxscaler} 
      x_{transformed} = \frac{x - x_{min}}{x_{max} - x_{min}}
\end{equation} for each of the datapoints \cite{JOJ-ENP}.
This ensures that the different scales of the data do not influence the clustering result and therefore improves the total clustering performance \cite{GOG-WSI}.

% Was sollen die folgenden beiden sections vermitteln?
%  Wichtig: Vergleichen zwischen verschiedenen Methoden, aber nur bei großen Unterschieden.
%  - Grundlegende Vorstellung der Methoden der Papers
%   - Aus welcher Perspektive gehen die Paper an die jeweilige Fragestellung heran?
%   - Warum verwendet das jeweilige Paper diese Methoden (wie sinnvoll ist das im gegebenen Kontext)? 
%   - Was möchten die jeweiligen Paper erreichen?
%   - In welchem Kontext stehen die Paper (bezogen auf die research questions)?
%   - Wie wenden die Paper k-means an?
%   - Dabei nicht jede Methodik einzeln aufschlüsseln, eher zusammenfassen.
%   - Werden die Daten danach weiter verarbeitet (weiterführende Theorien, ...)?
\subsection{Methodology of the Research Questions}
\label{sec:methodology_of_the_research_questions}
Three papers provide findings impacting the given research questions.
Each paper has a different perspective, namely the perspective of the industry, private households, and energy load profiles of a specific country.
The following section describes the goals of each paper and how the research is conducted to achieve these goals.

\enquote{\textit{Big Data-Informed Energy Efficiency Assessment of China Industry Sectors based on K-Means Clustering}} by Liu et al. \cite{LIU-BDE} focuses on clustering Chinese industry sectors and distinct companies into three clusters, based on environmental perfomance derived metrics.
These three clusters are then used to identify and compare the most energy-intensive sectors and companies in big data masses.
Liu et al. apply k-means clustering on disclosed environmental performance data of 15.128 Chinese companies between 2000 and 2015.
By applying k-means clustering with a fixed value of $k=3$, the companies and industry sectors are divided into low-, mid-, and high-performing clusters according to the chosen metrics, for example, $SO_2$ emissions.

\enquote{\textit{Identifying Home System of Practices for Energy Use with K-Means Clustering}} by Malatesta et al. \cite{MAL-HBP} focuses on finding home systems of practices (HSOPs) by clustering energy load profiles of 39 private households in Blacktown, Australia.
Each cluster represents a home system of practices (HSOP), a habit or behavior that alters the standard load profile.
The clustering is executed on the total dataset and separately for each household, each cluster is then labeled as a distinct HSOP.
This allows the researchers to identify general trends from the total data clustering and precise comparisons between households by comparing the individual household clustering results.
By applying social theory and survey results distributed to the households' residents afterward it can be derived how behaviors and habits impact electricity consumption.

\enquote{\textit{Identification of Natural Disaster Impacted Electricity Laod Profiles with k-means Clustering Algorithm}} by Jessen et al. \cite{JES-IND} investigates the load profiles of Lombok, Indonesia between 2015 and 2022.
The data is clustered in a varying number of clusters.
The value of $k$ is chosen according to the given dataset using the elbow method.
These clusters are then algorithmically labeled as normal or abnormal, with abnormal being either natural disaster-impacted or electrical fault-impacted which means non-human-induced.
This shows how k-means helps in understanding patterns, trends and the consequences of different impacts on the electricity load profiles.

All of these papers contribute to understanding general electricity load profiles and how different impacts alter the consumption data.
The results are then used to conclude how the design of energy-saving incentives can be improved.


% Evaluation and Results section
\chapter{Findings}
\label{cha:findings}

\begin{itemize}
    \item Steps taken to arrive at the presented findings 
    \item research question is addressed
    \item present findings
    \item graphs and table support main findings 
    \item findings are presented without judgement
\end{itemize}


% Discussion section
\chapter{Discussion \& Conclusion}
\label{cha:discussion}

\begin{itemize}
    \item summarize results, highlight main points
    \item cover results in the context of original motivation and problem statement
    \item discuss how thesis addresses original questions
    \item discuss limitations of the thesis
    \item "so what" message is formulated (key take homes, how does it help reader in the future)
    \item potential next steps
\end{itemize}

% Outlook section

\section{Outlook}
\label{sec:outlook}
% Auf climate change eingehen
The main goal of the thesis is to use k-means to design energy-saving incentives and understand energy load patterns.
Still, there are ways to improve the research and the results in the future.

% Connect to the limitations
As mentioned in the limitations of the thesis, the reviewed papers are \textit{biased towards their energy source}.
To fully understand energy load patterns and design universally applicable energy-saving incentives, the research needs to be extended.
This can be done by including different cultural and geographical regions, different environmental conditions and different energy sources over a larger timespan.
Additionally, this will eliminate the bias towards the COVID-19 pandemic due to the larger timespan.
This implies the need for a larger dataset to be analyzed in the future.

Also, the \textit{correctness of the k-means algorithm's results} needs to be proven in further research since it was not communicated in the reviewed papers.
The algorithm has to be executed multiple times to ensure the correctness of the results.
Another possibility is to use different clustering algorithms like the introduced k-harmonic means or MAP-DP, comparing the results and reviewing their correctness manually or by using data analysis tools.
To verify that no contrary position can be proven by the clustering results, one can assume the contrary position to be true and check whether the clustering results can be applied in a way that proves the contrary position wrong.

Future research should also focus on \textit{applying higher-level models to the clustering results} to generate more knowledge out of the data.
Like Malatesta et al. \cite{MAL-HBP} did, the clustering results are just the first step in making data more available and answering the research questions.
Further research must focus on applying higher-level models to the clustering results.

Finally, \textit{testing different values for $k$} is essential for further research.
Jessen et. al \cite{JES-IND} fail to spot single outliers without a general trend in the data.
Ongoing research focussing on testing different values for $k$ can help to identify these outliers since the algorithm might spot and assign them to a single cluster.

% What can people do in the future 
In the future, further research with larger, more different datasets and different clustering algorithms is needed to fully understand and answer the research questions.

% Conclusion section
\section{Conclusion}
\label{cha:conclusion}
%//TODO conclusion mit discussion mergen, nicht mehr uterteilen
%//TODO Outlook am Schluss
% summarize results, highlight main points
% cover results in the context of original motivation and problem statement
% "so what" message is formulated (key take homes, how does it help the reader in the future)
% discuss how the thesis addresses original questions

% Hauptaussage: K-Means ist toll um Trends zu spotten oder erste Eigenschaften aus einer Datenmenge herauszulesen, 
% allerdings kann man diesen Algorithmus nicht blind auf jedes Problem anwenden


% Auf Motivation eingehen
This review aimed to tackle climate change's root causes in the energy sector.
This meant understanding general electricity load patterns and using this knowledge to improve the design of energy-saving incentives.
K-means clustering was introduced as a tool for tackling these problems by applying it to different datasets in the context of two research questions.
The first research questions aimed to generally understand electricity load patterns by identifying patterns and characteristics in a given dataset.
Question two aimed to design energy-saving incentives in the context of private housing and industry.

Both research questions showed, that k-means is a powerful tool for spotting trends and characteristics in data.
It helps to identify the most energy-intensive sectors and companies in big data masses, which helps provide targeted support and reduce information asymmetry \cite{LIU-BDE}.
It identifies different consumer routines and habits in the consumption data which can lay the foundation for applying more advanced methods and theory, thus revealing even more knowledge \cite{MAL-HBP}.
Therefore, k-means is a powerful tool for finding trends, identifying characteristics and revealing hidden knowledge in big data masses.

Also, the flaws of the algorithm were discussed.
It was shown, that k-means cannot spot single outliers in big datasets \cite{JES-IND}.
Therefore, the detection of single earthquakes failed despite finding general load profile trends and seasonal patterns \cite{JES-IND}.
Furthermore, due to the algorithm being sensitive to initial starting conditions and the chosen value for $k$, multiple runs with different initial cluster centers are needed to obtain the optimal cluster result \cite{JAI-DCB}, \cite{EZU-CPF}, \cite{BAR-LVG}.
Also, result manipulation is possible due to the algorithm not always finding obvious cluster structures \cite{BOU-RPK}.
This makes the algorithm computationally expensive and time-consuming, despite the actual algorithm being simple and fast.

% Summarize results in three sentences
% Include Key message and take homes
Through this, k-means makes data more publicly available and understandable, which is a key factor for efficient design and decision processes in creating energy-saving incentives.
Due to its simplicity, it can be applied to a wide range of problems, finding general trends and patterns.
However, it is not a one-size-fits-all solution and should be used with caution, as it can be computationally expensive and time-consuming.

The choice of the best clustering algorithm should be guided by the given requirements of the data and the goals of the analysis \cite{COL-ALT}.
K-means clustering is among the most popular clustering algorithms.
This popularity is justified by its simplicity, efficiency in handling big datasets and effectiveness while delivering good results.
One must weigh the balance between efficiently identifying general trends and patterns as suggested by k-means, and more explicit but less effective algorithms.
%//TODO: Eine "so what" message formulieren = How does the written report help the reader in the future?
So, k-means proved as the ideal tool for the given research questions, it is a great starting point in understanding general electricity load patterns and designing energy-saving incentives.


% References section (Bibliography)
\bibliography{references} % Include your bibliography file (e.g., references.bib)

\end{document}
